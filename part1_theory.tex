\documentclass[a4paper, 11pt, english]{article}
	\usepackage{listings, babel, blindtext}
	\author{Kim Rune Solstad, TDT4200, NTNU}
	\title{Assignment 2, MPI}

	\lstset{language=c}

\begin{document}
\maketitle
\section{Problem 1}
\subsection{a)}
Explain the difference between MPI\_Send(), MPI\_Isend() and MPI\_Ssend().
\\
\\
The three functions mentioned are mpi send functions with different modes. Different modes will give the program different guarantees. The four send modes in MPI are:
\begin{enumerate}
	\item Standard
	\item Synchronized
	\item Bufferred
	\item Ready
\end{enumerate}

MPI\_Send() is the standard mode. It will send the message absent any hazzles or guaranties. MPI\_Ssend() is the send function that belongs to the synchronized mode. MPI\_Ssend() will complete the transmission when it has been received at the other end. All of these functions have non-blocking alternatives. By blocking, we mean that execution will be suspended until the message buffer is safe to use. MPI\_Isend is the non-blocking alternative to MPI\_Send(). 
\subsection{b)}
What is a communicator in MPI? What is the purpose of a cartesian communicator?
\\\\
A communicator is a collection of processes that can send messages to each other. The cartesian communicator has a topology of a multi-dimensional array.
\section*{Ref}
\begin{enumerate}
	\item www.ncsa.illinois.edu/UserInfo/Resources/Hardware/CommonDoc/MessPass/MPIBlock.html
\end{enumerate}

\end{document}
